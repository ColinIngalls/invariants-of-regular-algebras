
\documentclass{article}
\usepackage{amsmath}
\newtheorem{proposition}{Proposition}
\newtheorem{theorem}{theorem}
\begin{document}

Let $k$ be a field and let $q_{ij} \in k$ for $1 \leq i < j \leq n$ and
for convenience, when $q_{ij} \ neq 0,$ we set $q_{ji}=q_{ij}^{-1}$ and $q_{ii}=1$ and write $Q  =(q_{ij})$
be the multiplicately skew-symmetric matrix.
Let $S=k\langle x_1,\ldots,x_n\rangle$.
Let $I$ be the two sided ideal of $S$ generated by
$$x_jx_i - q_{ij} x_ix_j\quad \quad 1 \leq i < j \leq n.$$
We let $A = A_Q$ be $S/I$.
Note that $A$ has a $k$-basis of ordered monomials and so we have
$$ A \simeq \bigoplus_{i_1\geq 0,\cdots, i_n \geq 0} k x_1^{i_1}\cdots x^{i_n}_n$$
as vector spaces.

We will consider first order deformations of $A = S/I$.
We consider $a_{ij} \in A$ with $1 \leq i < j \leq n$.
Let $k_1 = k[\varepsilon]/(\varepsilon^2)$ and write $\alpha_{ij} = \varepsilon a_{ij}$.
$$x_jx_i - q_{ij} x_ix_j  - \alpha_{ij} \quad \quad 1 \leq i < j \leq n.$$
We obtain the following family of algebras
$$A_1 = k_1\langle x_1,\ldots,x_n \rangle/I_1.$$

\begin{proposition}
  The algebra $A_1$ is a {\it first order deformation} of $A$ if and only one of the three equivalent conditions hold.
  \begin{enumerate}
    \item $A_1$ is a flat $k_1$ algebra with a fixed isomorphism $A_1/(\varepsilon) \simeq A$.
    \item $\varepsilon A_1 \simeq A$ as $k$-vector spaces.
    \item $(a_{ij})$ are a Hocschild cocyle giving a class in $HH^2(A)$.
    \item If $k>j>i$ when we reduce $(x_kx_j)x_i = x_k(x_jx_i)$ both ways to ordered monomials, we get the same answer.
  \end{enumerate}
\end{proposition}

We begin by computing the fourth condition above.
\begin{eqnarray*}
  x_k(x_jx_i) 
& = & x_k(q_{ij} x_ix_j + \alpha_{ij}) = q_{ij} x_kx_ix_j+ x_k \alpha_{ij} \\
& = & q_{ij} (q_{ik}x_ix_k + \alpha_{ik})x_j +x\alpha_{ij}\\
& = & q_{ij}q_{ik}x_ix_kx_j +q_{ij}\alpha_{ik} x_j + x_k \alpha_{ij}\\
& = & q_{ij}q_{ik}q_{jk} x_ix_jx_k + q_{ik}q_{ik}x_i\alpha_{jk} + q_{ij}\alpha_{ik}x_j + x_k\alpha_{ij}
\end{eqnarray*}
\begin{eqnarray*}
(x_kx_j)x_i
= (q_{jk} x_jx_k + \alpha_{jk})x_k = q_{jk} x_jx_kx_i+ \alpha_{jk}x_i \\
= q_{jk}x_j(q_{ik}x_ix_k+\alpha_{ik}) + \alpha_{jk}x_i \\
= q_{jk}q_{ik}x_jx_ix_k+q_{jk}x_j\alpha_{ik}+\alpha_{jk}x_i \\
= q_{jk}q_{ik}q_{ij}x_ix_jx_k + q_{jk}q_{ik} \alpha_{ij} x_k +q_{jk}x_j\alpha_{ik} \alpha_{jk}x_i
\end{eqnarray*}

Hence we have a first order deformation if and only if
$$q_{ij}q_{ik}x_i\alpha_{jk}+q_{ij}\alpha_{ik}x_j+x_k\alpha_{ij} \\
= q_{jk}q_{ik} \alpha_{ij}x_k + q_{jk}x_j\alpha_{ik} +\alpha_{jk}x_i$$

We will solve these equations for graded deformations, i.e.~when $\alpha_{ij} \in \varepsilon A_2$ are quadratic.  This could be done for choices of degree other than two.

So we let
$$\alpha_{ij} = \sum_{1 \leq \ell \leq m \leq n} \alpha_{ij}^{\ell m} x_\ell x_m$$
be arbitrary quadratic elements of $A$.  We look at the above condition for these $\alpha$.
So we have 
$$ \sum(q_{ij}q_{ik}x_i\alpha_{jk}^{\ell m} x_\ell x_m+q_{ij}\alpha_{ik}^{\ell m} x_\ell x_mx_j+x_k\alpha_{ij}^{\ell m} x_\ell x_m) \\
= \sum( q_{jk}q_{ik} \alpha_{ij}^{\ell m} x_\ell x_mx_k + q_{jk}x_j\alpha_{ik}^{\ell m} x_\ell x_m +\alpha_{jk}^{\ell m} x_\ell x_mx_i)$$
This gives cubic expressions in $A$, which we can separate in to different equations, for example:
$$(q_{ij} q_{ik} \alpha_{jk}^{ii} -\alpha_{jk}^{ii}) x_i^3 = 0.$$
On examining the coefficients of different monomials, we obtain the following
equations:
$$ (q_{ij}q_{ik}-q_{i\ell})\alpha_{jk}^{i\ell} = 0 \quad  \{ j,k \} \cap \{i,\ell\} = \emptyset, \quad  j<k, \quad i \leq \ell$$
  $$ q_{ij}(q_{ik}-q{i\ell}) \alpha_{jk}^{j\ell} + q_{ij}(q_{j\ell}-q_{jk}) \alpha_{ik}^{i\ell} = 0 \quad \quad |\{ i,j,k \}| =3.$$


We note that these conditions are mostly independent.  The first set of equations is completely independent from each other and the second set of equations.
The second set of equations decomposes into sets of three equations
for each ordered pair $k,\ell$ with $k \neq \ell.$
Hence we can combine them into a matrix equation for $i<j<k$ we have
$$\begin{pmatrix}
  q_{jk} & & \\
    & q_{ik} & \\
    & & q_{ij}
\end{pmatrix}
\begin{pmatrix}
  & q_{k \ell}-q_{kk} & q_{j\ell} - q_{jk} \\
  q_{kk} - q_{k \ell} & 0 & q_{i\ell} - q_{ik} \\
  q_{jk} - q_{j \ell} & q_{ik} - q_{i\ell} & 0
\end{pmatrix}
$$










We first study the deformations for generic $Q=(q_{ij})$.
We will determine later that the exact conditions we need are:
$$q_{ij} \neq 0 \quad \quad 1 \leq i < j \leq n.$$
$$q_{ij}q_{ik} \neq q_{i\ell} \quad \quad \{ j,k \} \cap \{i,\ell\} = \emptyset $$
$$ Q \notin V(q_{})$$
Next, we will discuss infinitesimal automorphisms of first order deformations.  Hochschild cohomology captures infinitesimal deformation up infitesimal isomorphism.  Let $d_i \in A$ for $1 \leq i \leq n$ and write $\delta_i = \varepsilon d_i$.  We can carry out an infinitesimal change of coordinates that will change our presentation of $A_1$.
Consider
$$x_i \mapsto x_i+\delta_i.$$
We can associate a derivation $\delta$ of $S$ to this information by
$$x_i \mapsto d_i$$ and extending linearly and by the Leibniz rule.
The effect of an infinitesimal change of coordinates on relations is determined
by $$ r_{ij} \mapsto \delta(r_{ij}).$$
In particular for our case, we have
$$ x_jx_i - q_{ij} x_ix_j \mapsto \delta_jx_i+x_j\delta_i -q_{ij} (\delta_i x_j+
x_i\delta_j).$$
This changes the relations in the following way:
\begin{proposition}
Two first order deformations $A_1$ and $A_1'$ determined by
$(a_{ij})$ and $(a'_{ij})$ are infinitesimally isomorphic
  if and only if
  \begin{enumerate}
  \item There is a commutative diagram
    $$A_1 \to A \\
    A_1' \to A $$
\item There are $d_i \in A$ so that
  have $$
  a_{ij} = a_{ij}' + d_j x_i + x_j d_i -q_{ij}(d_i x_j + x_id_j).$$
  \end{enumerate}
\end{proposition}



\end{document}
