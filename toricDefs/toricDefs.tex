
\documentclass{article}
\usepackage{amsmath,tikz-cd}
\newtheorem{proposition}{Proposition}
\newtheorem{definition}{Definition}
\newtheorem{theorem}{Theorem}
\newtheorem{question}{Question}

\DeclareMathOperator{\PGL}{PGL}
\DeclareMathOperator{\Gr}{Gr}
\DeclareMathOperator{\Ext}{Ext}
\DeclareMathOperator{\Aut}{Aut}

\begin{document}


\section{Deformations of Toric Algebras}
Let $k$ be a field and let $q_{ij} \in k$ for $1 \leq i < j \leq n$ and
for convenience, we set $q_{ii}=0$ and when $q_{ij} \neq 0,$ we set $q_{ji}=q_{ij}^{-1}$.  We also write $Q  =(q_{ij})$
be the multiplicately skew-symmetric matrix.
Let $S=k\langle x_1,\ldots,x_n\rangle$.
Let $I$ be the two sided ideal of $S$ generated by
$$x_jx_i - q_{ij} x_ix_j\quad \quad 1 \leq i < j \leq n.$$
We let $A = A_Q$ be $S/I$.  This is a family of algebras of dimension $\binom{n}{2}.$
Note that $A$ has a $k$-basis of ordered monomials and so we have
$$ A \simeq \bigoplus_{i_1\geq 0,\cdots, i_n \geq 0} k x_1^{i_1}\cdots x^{i_n}_n$$
as vector spaces.  Note that the Hilbert Series of $A$ is $H_A(t) = 1/(1-t)^n$ and if $q_{ij} \neq 0$ for all $i<j$ then $A_Q$ is an AS-regular algebra.
We will compute the Hochschild cohomology
$HH^2(A)_0$ using the degree zero part of the complex
$$ A(1)^n_0 \stackrel{d_1}{\to} A(2)^{\binom{n}{2}}_0 \stackrel{d_2}{\to}
  A(3)^{\binom{n}{3}}_0 $$
  We interpret $HH^2(A)_0$ as graded first order deformations, and we first compute the kernel of $d_2$.
  
To compute  first order deformations of $A = S/I$, we perturb the relations.
We consider $a_{ij} \in A$ with $1 \leq i < j \leq n$.
Let $k_1 = k[\varepsilon]/(\varepsilon^2)$ and write $\alpha_{ij} = \varepsilon a_{ij}$.
$$x_jx_i - q_{ij} x_ix_j  - \alpha_{ij} \quad \quad 1 \leq i < j \leq n.$$
We obtain the following family of algebras
$$A_1 = k_1\langle x_1,\ldots,x_n \rangle/I_1.$$

\begin{proposition}
  The algebra $A_1$ is a {\it first order deformation} of $A$ if and only one of the three equivalent conditions hold.
  \begin{enumerate}
    \item $A_1$ is a flat $k_1$ algebra with a fixed isomorphism $A_1/(\varepsilon) \simeq A$.
    \item $\varepsilon A_1 \simeq A$ as $k$-vector spaces.
    \item $(a_{ij})$ are a Hocschild cocyle giving a class in $HH^2(A)$.
    \item If $k>j>i$ when we reduce $(x_kx_j)x_i = x_k(x_jx_i)$ both ways to ordered monomials, we get the same answer.
  \end{enumerate}
\end{proposition}

We begin by computing the fourth condition above.
\begin{eqnarray*}
  x_k(x_jx_i) 
& = & x_k(q_{ij} x_ix_j + \alpha_{ij}) = q_{ij} x_kx_ix_j+ x_k \alpha_{ij} \\
& = & q_{ij} (q_{ik}x_ix_k + \alpha_{ik})x_j +x_k\alpha_{ij}\\
& = & q_{ij}q_{ik}x_ix_kx_j +q_{ij}\alpha_{ik} x_j + x_k \alpha_{ij}\\
& = & q_{ij}q_{ik}q_{jk} x_ix_jx_k + q_{ij}q_{ik}x_i\alpha_{jk} + q_{ij}\alpha_{ik}x_j + x_k\alpha_{ij}
\end{eqnarray*}
\begin{eqnarray*}
(x_kx_j)x_i
& = & (q_{jk} x_jx_k + \alpha_{jk})x_i = q_{jk} x_jx_kx_i+ \alpha_{jk}x_i \\
& = & q_{jk}x_j(q_{ik}x_ix_k+\alpha_{ik}) + \alpha_{jk}x_i \\
& = & q_{jk}q_{ik}x_jx_ix_k+q_{jk}x_j\alpha_{ik}+\alpha_{jk}x_i \\
& = & q_{jk}q_{ik}q_{ij}x_ix_jx_k + q_{jk}q_{ik} \alpha_{ij} x_k +q_{jk}x_j\alpha_{ik} + \alpha_{jk}x_i
\end{eqnarray*}

Hence we have a first order deformation if and only if
$$q_{ij}q_{ik}x_i\alpha_{jk}+q_{ij}\alpha_{ik}x_j+x_k\alpha_{ij} \\
= q_{jk}q_{ik} \alpha_{ij}x_k + q_{jk}x_j\alpha_{ik} +\alpha_{jk}x_i$$

We will solve these equations for graded deformations, i.e.~when $\alpha_{ij} \in \varepsilon A_2$ are quadratic.  This could be done for choices of degree other than two.

So we let
$$\alpha_{ij} = \sum_{1 \leq \ell \leq m \leq n} \alpha_{ij}^{\ell m} x_\ell x_m$$
be arbitrary quadratic elements of $A$.  We look at the above condition for these $\alpha$.
So we have 
$$ \sum(q_{ij}q_{ik}x_i\alpha_{jk}^{\ell m} x_\ell x_m+q_{ij}\alpha_{ik}^{\ell m} x_\ell x_mx_j+x_k\alpha_{ij}^{\ell m} x_\ell x_m) = $$ $$\sum( q_{jk}q_{ik} \alpha_{ij}^{\ell m} x_\ell x_mx_k + q_{jk}x_j\alpha_{ik}^{\ell m} x_\ell x_m +\alpha_{jk}^{\ell m} x_\ell x_mx_i)$$
This gives cubic expressions in $A$, which we can separate in to different equations, for example:
$$(q_{ij} q_{ik} \alpha_{jk}^{ii} -\alpha_{jk}^{ii}) x_i^3 = 0.$$
On examining the coefficients of different monomials, we obtain the following
equations:
$$ (q_{ij}q_{ik}-q_{i\ell})\alpha_{jk}^{i\ell} = 0 \quad  \{ j,k \} \cap \{i,\ell\} = \emptyset, \quad  j<k, \quad i \leq \ell$$
  $$ q_{ij}(q_{ik}-q{i\ell}) \alpha_{jk}^{j\ell} + q_{ij}(q_{j\ell}-q_{jk}) \alpha_{ik}^{i\ell} = 0 \quad \quad |\{ i,j,k \}| =3.$$


We note that these conditions are mostly independent.  The first set of equations is completely independent from each other and the second set of equations.
The second set of equations decomposes into sets of three equations
for each ordered pair $k,\ell$ with $k \neq \ell.$
Hence we can combine them into a matrix equation for $i<j<k$ we have
$$\begin{pmatrix}
  q_{jk} & & \\
    & q_{ik} & \\
    & & q_{ij}
\end{pmatrix}
\begin{pmatrix}
  0 & q_{k \ell}-q_{kk} & q_{jk} - q_{j\ell} \\
  q_{kk} - q_{k \ell} & 0 & q_{i\ell} - q_{ik} \\
  q_{j\ell} - q_{jk} & q_{ik} - q_{i\ell} & 0
\end{pmatrix}
\begin{pmatrix}
  \alpha^{ik}_{i\ell} \\ \alpha^{jk}_{j\ell} \\  \alpha^{kk}_{k\ell}
  \end{pmatrix} = 0
$$

Recall that a skew symmetric matrix always has even rank, so we always have a solution
$$\begin{pmatrix}
  \alpha^{ik}_{i\ell} \\ \alpha^{jk}_{j\ell} \\  \alpha^{kk}_{k\ell}
\end{pmatrix}
\in k
\begin{pmatrix}
  q_{i\ell} - q_{ik} \\ q_{j\ell} - q_{jk} \\ q_{k\ell} - q_{kk}
\end{pmatrix}.$$
These are all solutions when the matrix has rank two, otherwise the matrix must have rank zero, so $$ q_{i\ell} = q_{ik},q_{j\ell} = q_{jk}, q_{k\ell} = q_{kk}$$
and there are no conditions on the corresponding  $\alpha^{ik}_{i\ell},\alpha^{jk}_{j\ell},\alpha^{kk}_{k\ell}.$  Hence we can conclude the following result.

\begin{definition}
  $Q$ is {\it generic} if 
  $$q_{ij} \neq 0 \quad i<j$$
  $$q_{ij}q_{ik} \neq q_{i\ell} \quad j<k, i \leq \ell.$$
  $$ Q \notin V(q_{i\ell} = q_{ik},q_{j\ell} = q_{jk}, q_{k\ell} = q_{kk}).$$
\end{definition}

\begin{proposition}
  Let $A_1$ be determined by the $(\alpha_{ij})$ then if $Q$ is generic $A_1$ is a flat deformation if and only if the $(\alpha_{ij})$ satisfy
  $$ \alpha_{jk}^{i\ell} = 0 \quad \quad j < k, i \leq \ell,$$
$$\begin{pmatrix}
  \alpha^{ik}_{i\ell} \\ \alpha^{jk}_{j\ell} \\  \alpha^{kk}_{k\ell}
\end{pmatrix}
\in k
\begin{pmatrix}
  q_{i\ell} - q_{ik} \\ q_{j\ell} - q_{jk} \\ q_{k\ell} - q_{kk}
\end{pmatrix}.$$
\end{proposition}

\begin{proposition}
  If $Q$ above is not generic, then we have further deformations as follows:
  \begin{itemize}
    \item If $q_{ij}q_{ik} = q_{i\ell}$ then we allow $\alpha_{ik}^{i\ell} $ to be free.
    \item If $q_{ij} = 0$ then we allow $\alpha_{jk}^{j\ell}, \alpha_{ik}^{i\ell}$ to be free.
    \item If $q_{i\ell} = q_{ik},q_{j\ell} = q_{jk}, q_{k\ell} = q_{kk}$
      then we allow $\alpha^{ik}_{i\ell} , \alpha^{jk}_{j\ell} ,  \alpha^{kk}_{k\ell}$ to be free.
      \end{itemize}
\end{proposition}

\begin{question}
  Does the family $A_Q$ meet other families of AS-regular algebras along the loci described above?
  \end{question}


Next, we will discuss infinitesimal automorphisms of first order deformations.  Hochschild cohomology captures infinitesimal deformation up infitesimal isomorphism.  Let $d_i \in A$ for $1 \leq i \leq n$ and write $\delta_i = \varepsilon d_i$.  We can carry out an infinitesimal change of coordinates that will change our presentation of $A_1$.
Consider
$$x_i \mapsto x_i+\delta_i.$$
We can associate a derivation $\delta$ of $S$ to this information by
$$x_i \mapsto d_i$$ and extending linearly and by the Leibniz rule.
The effect of an infinitesimal change of coordinates on relations is determined
by $$ r_{ij} \mapsto \delta(r_{ij}).$$
In particular for our case, we have
$$ x_jx_i - q_{ij} x_ix_j \mapsto \delta_jx_i+x_j\delta_i -q_{ij} (\delta_i x_j+
x_i\delta_j).$$
This changes the relations in the following way:
\begin{proposition}
Two first order deformations $A_1$ and $A_1'$ determined by
$(a_{ij})$ and $(a'_{ij})$ are infinitesimally isomorphic
  if and only if
  \begin{enumerate}
  \item There is a commutative diagram
   $$\begin{tikzcd}[column sep=small]
A_1 \arrow{r}  \arrow{rd} 
  & A_1' \arrow{d} \\
    & A
\end{tikzcd}$$
\item There are $d_i \in A$ so that
  have $$
  a_{ij} = a_{ij}' + d_j x_i + x_j d_i -q_{ij}(d_i x_j + x_id_j).$$
  \end{enumerate}
\end{proposition}

We consider $\delta_i = \sum^j_i x_i$ and consider the infintesimal deformations we obtain by infintesimal change of coordinates, i.e. trivial deformations.
We see the relations become
$$(x_j+\delta_j) (x_i+\delta_i) = q_{ij}(x_i + \delta_i)(x_j+\delta_j)$$
$$x_jx_i + \delta_jx_i + x_j\delta_i =q_{ij}(x_ix_j + \delta_ix_j + x_i\delta_j)$$
$$x_jx_i = q_{ij} x_ix_J+q_{ij}\delta_ix_j + x_i\delta_j - \delta_jx_i + x_j\delta_i$$
and so
$$\alpha_{ij} =  q_{ij}\delta_ix_j + x_i\delta_j - \delta_jx_i + x_j\delta_i$$
$$ \alpha_{ij} = q_{ij}\delta_i^kx_kx_j + x_i\delta_j^\ell x_\ell - \delta_j^\ell x_\ell x_i + x_j\delta_i^k x_k.$$
$$ = \delta_j^k(q_{ik}-q_{ij})x_ix_k + \delta_i^\ell (q_{\ell j} -q_{ij})x_\ell x_j.$$

\begin{theorem}
  Let $A_Q$ be generic, then any deformation of $A_Q$ lies in the same family and so $A_Q$ forms a component of the moduli space of AS-regular algebras of dimension four.  This component possibly meets other components of the moduli space when we have some 
  $$i<j \quad q_{ij} = 0$$
  or $$i<j,k,\ell \quad \quad q_{ij}q_{ik} = q_{i\ell}$$
  or
 $$q_{i\ell} = q_{ik},q_{j\ell} = q_{jk}, q_{k\ell} = q_{kk}$$
\end{theorem}


\begin{question}
Can we find other components of the moduli space that meet $A_Q$ in the loci described above?
\end{question}
\begin{question}
  Does this theorem work for any $n$, not just $n=4$.
\end{question}
\begin{question}
  The families of algebras described in Pym's thesis are all components.  How do they meet other components and each other?
\end{question}

\section{The moduli space of AS-regular Algebras}
In this section we discuss the moduli space of AS-regular algebras and some rigidifications. We will obtain a closed variety and so we will allow algebras that are possibly not regular.
Let $A$ be generated by a vector space of dimension four $V$.
Then $A$ has six quadratic relations.


This gives a point in
$$A \in \Gr(6, V\otimes V).$$  Alternatively, we can consider the Koszul dual in
$$A^! \in \Gr(10, V^*\otimes V^*).$$  Imposing
$$\dim A_d = \binom{3+d}{d}$$
is a closed condition for each $d$.  Since the intersection of closed spaces is closed, we see that
$$ A_4 := \{ A \in \Gr(6,V\otimes V) \,|\, H_A(t) = 1/(1-t)^4\}
\subset \Gr(6,V\otimes V).$$ is a closed subvariety.
The algebra defined by $x_jx_i = 0$ for $i<j$ is in $M_4$ and is not AS-regular, so we know that $M_4$ is larger than the set of AS-regular algebras.
\begin{question} Is the generic point of each component of $M_4$ regular?
  \end{question}
Changing coordinates gives an action of
$\PGL_4V$ on this variety and the moduli (stack) space
of algebras with this Hilbert series is
$$ M_4 = A_4/\PGL_4V \subset \Gr(6,V\otimes V)/\PGL V.$$

Note that the stabilzer of a point $A_0 \in M_4$ is $\Aut A/k^*$.
Recall we always have that $k^* \subseteq \Aut A \subseteq \GL V.$

Since deformation are flat, they preserve the Hilbert Series.  So Hochschild cohomlogy measures the tangent spaces of $M_4$.  More precisely, let $A_0 \in M_4$,
then $$HH^2(A)_0 = T_{A_0} M_4$$ as $\Aut A/k^*$ equivariant vector spaces.

We can rigidify these spaces.  Fist recall there are standard open subsets of the Grassmanian, with the simplest being the span of 
$$( I_6 |P)$$ where
$P$ is an arbitrary $6 \times 10$ matrix.



We define $A^! = A^\perp$ and if
$A$ is Koszul then we know futher that $A^! \simeq \Ext^*_A(k,k).$
and $\dim A^!_d = \binom{4}{d}$.  We could impose these closed conditions on $M_4$ to obtain a possibly smaller variety, whose algebras are more likely to be AS-regular.


Here is another approach to the moduli space that is somewhat easier.
We consider the moduli space of Frobenius graded $k$-algebras with Hilbert series $H_A(t) = (1+t)^4$. Let $A$ be such an algebra.

We can take generators in a vector space $V$ of dimension four.  Then
we have an ideal generated by 10 quadratics $I_2 \in V\otimes V.$
So we begin with a subset of $\Gr(10,V \otimes V).$
To obtain the right dimension for the space of cubics, we require that
$$\dim I_2 V + V I_2 = 60.$$
So we have a linear map
$$m: (I_2 \otimes V) \oplus (V \otimes I_2) \to V\otimes V \otimes V.$$
So we can let
$$A_4 = \{ m { as above } \,|\, \rank m = 60. \}.$$


\section{Clifford Algebras}
Let $V$ be a vector space of dimension $n$.
For graded Clifford algebras we have subspace $R$ of relations in $\Sym^2 V$
of dimension $\binom{n}{2}$.
This gives an algebra $k\langle V \rangle/R$.
\end{document}

